%% Packages
\documentclass[12pt,a4paper]{report}
\usepackage[T1]{fontenc}
\usepackage[utf8]{inputenc}
\usepackage{minitoc}
\usepackage[pdftex]{graphicx}
\usepackage[a4paper,left=2cm,right=2cm,top=2cm,bottom=2cm]{geometry}begin{document}
\usepackage[frenchb]{babel}
\usepackage{libertine}
\setcounter{tocdepth}{1}
\setcounter{minitocdepth}{3}

\begin{titlepage}
	\begin{center}

		\includegraphics[scale=1.8]{garde.jpg}~\\[1.5cm]
		\textsc{\LARGE CY-TECH MI GROUPE 4}\\[2cm]	
		
		\textsc{\Large RAPPORT DE PROJET BUREAU D'ETUDES}\\[1.5cm]

    % Title
    \rule{\linewidth}{0.4pt} \\[0.4cm]
    {\huge \bfseries Objectif : déterminer le temps de vie du polonium 212 \\[0.4cm]}
    \rule{\linewidth}{0.4pt} \\[2cm]
    \textsc{\Large DURECU Clément}\\[1.5cm]
    \textsc{\Large GOBE-TRUONG Evan}\\[1.5cm]
    \textsc{\Large PROVENT Amaury}\\[1.5cm]
    \textsc{\Large LLINARES UBERTALLI Tom}\\[1.5cm]
\end{titlepage}

\begin{document}
%% Document

   \dominitoc
   \tableofcontents
   \chapter{Le projet}
      \section{Sujet}
      	\textsc{Le  projet  consiste  à  déterminer  le  temps  de  demi-vie  du  polonium 212 qui  se désintègre en émettant une particule $\alpha$ dont l’énergie est comprise entre 8 MeV et 9 MeV. Dans un premier temps, vous utiliserez le programme qui vous donnera une valeur de la probabilité de sortie de la particule $\alpha$ en dehors du noyau et à l’aide d’un modèle que vous aurez élaboré, vous en déduirez un ordre de grandeur du temps de demi-vie. Dans un second temps vous comparerez la valeur ainsi obtenue aux valeurs expérimentales et celle déterminée à l’aide de la formule de Gamow.}
      \section{Lien avec le cours}
   \chapter{Réalisation}
   		\section{Première étape}
   		\section{Deuxième étape}
\end{document}